%!TEX root=../document.tex

\section{Ergebnisse}
\label{sec:Ergebnisse}

\subsection{Theoretische Fragestellungen}
\subsubsection{Nutzung von GPUs in normalen Desktop-Anwendungen}
\textbf{Einführung}\\
Die Nutzung von GPUs in Desktop-Anwendungen ist meistens unter dem Begriff ''Hardware acceleration'' (Hardwarebeschleunigung) bekannt.\\
Hardwarebeschleunigung entlastet die CPU durch Delegation rechenintensiver Aufgaben an für diese Aufgaben optimierte Hardware. In den meisten Fällen ist dies die GPU. \cite{hardwareacc_streamingmedia} \\\\
Der Vorteil von GPUs gegenüber CPUs ist die Nebenläufigkeit. CPUs sind sequentiell ausführende Prozessoren, während GPUs auf nebenläufige Ausführung spezialisiert sind. \cite[S. 2-3]{programminggpgpu_kirk} \\
Für bestimmte, rechenintensive Aufgaben ist es daher sinnvoll, die GPU anstelle der CPU zu verwenden. Allerdings sollte beachtet werden, dass die Delegation an die GPU durchaus aufwändig sein kann.\\
Beispielsweise ist das Kopieren von Ressourcen ein Faktor, welcher berücksichtigt werden sollte.\\\\
\textbf{Anwendungen}\\
Hier einige Anwendungen, welche Programmteile auf der GPU ausführen:
\begin{itemize}
\item Microsoft Office
\item VLC
\item Verschiedene Browser (Besonders für HTML5)
\end{itemize}
Zusätzlich ist es meistens möglich, die Hardwarebeschleunigung manuell ein- bzw. auszuschalten.
\newpage
\subsubsection{IDEs und Programmiersprachen}
\textbf{OpenCL}\\
Eine OpenCL-Applikation besteht aus 2 Bestandteilen - dem Host und den ''Compute Devices''. Die Compute-Devices stellen CPUs, GPUs oder andere Prozessortypen dar. Auf den Devices findet die eigentliche Berechnung statt. Der Host ist dafür zuständig, die Aufgaben an die Devices zu verteilen. \cite{openclprogramming_munshi} \\\\
Der Host-Code kann dabei in verschiedenen Programmiersprachen wie C++ oder Java programmiert werden. Der Code für die Compute-Devices (Kernels) wird allerdings in der OpenCL Programming Language, auch OpenCL C genannt, programmiert. \cite{openclprogramming_munshi}\\\\
Als Entwicklungsumgebung kann eine beliebige IDE für die jeweilige Sprache genutzt werden. Um besondere Funktionen für die nebenläufige Programmierung zu nutzen, bietet OpenCL \\''OpenCL Studio'' an. \cite[S. 12]{openclstudio} \\\\
\textbf{CUDA}\\
Eine CUDA-Anwendung besteht ebenfalls aus einem Host und mehreren Devices. Der Device-Code wird dabei ebenfalls in einer C-ähnlichen Sprache verfasst. Für den Host-Code existieren ebenfalls diverse Bindings. \cite[S. 41]{programminggpgpu_kirk} Als speziell angepasste ''IDE'' stellt Nvidia das Nsight Eclipse Plugin zur Verfügung. \cite{nsighteclipse}
\subsubsection{Bestehende Programme auf GPUs nutzen}
Prinzipiell ermöglichen es bestimmte Tools, bestehenden Code in OpenCL zu übersetzen. Allerdings können die Ergebnisse wenig zufriedenstellend ausfallen. In den meisten Fällen ist es sinnvoller, den OpenCL C Code selber zu schreiben. Dies macht es möglich, den Code entsprechend zu parallelisieren und zu optimieren. Das Übersetzen von bestehendem Code in OpenCL C ist oft nicht ausreichend um eine Performanceoptimierung zu erzielen.\\\\
\textbf{Aparapi}\\
Aparapi ist ein Tool, dass Java Bytecode während der Laufzeit in OpenCL C übersetzt und auf der GPU ausführt. Wenn die Übersetzung wegen eines Fehlers nicht möglich ist wird der Code in Thread Pools auf der CPU ausgeführt.\cite{aparapi}

\subsubsection{Transcompiler}

\newpage
\subsection{Bruteforce}
\subsubsection{Beschreibung}
\subsubsection{Systemdaten}
\begin{itemize}
\item Intel Core i7-3635QM CPU 2.40GHz
\item 8 GB DDR3-RAM
\item Samsung MZ-75E500B/EU 850 EVO 2,5 Zoll 500GB SSD (SATA III)
\item Windows 10 Pro 64-Bit
\item AMD Radeon R9 M200X Series 2048 MB
\end{itemize}
\subsubsection{Durchführung}
Es wurden jeweils 100 Durchläufe durchgeführt.
Die Ergebnisse sehen folgendermaßen aus:

